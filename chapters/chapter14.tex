\chapter{Simple Network Time Protocol (SNTP)}

Primary servers and clients complying with a subset of NTP, called
the Simple Network Time Protocol (SNTPv4) [RFC4330], do not need to
implement the mitigation algorithms described in Section 9 and
following sections.  SNTP is intended for primary servers equipped
with a single reference clock, as well as for clients with a single
upstream server and no dependent clients.  The fully developed NTPv4
implementation is intended for secondary servers with multiple
upstream servers and multiple downstream servers or clients.  Other
than these considerations, NTP and SNTP servers and clients are
completely interoperable and can be intermixed in NTP subnets.

An SNTP primary server implementing the on-wire protocol described in
Section 8 has no upstream servers except a single reference clock.
In principle, it is indistinguishable from an NTP primary server that
has the mitigation algorithms and therefore capable of mitigating
between multiple reference clocks.

Upon receiving a client request, an SNTP primary server constructs
and sends the reply packet as described in Figure 31.  Note that the
dispersion field in the packet header must be updated as described in
Section 5.

\begin{table}[htb]
\center
\begin{tabular}{c | c | c}
Association Mode & Assoc. Mode Value & Packet Mode Value \\
\hline
\hline
Symmetric Active & 1 & 1 or 2 \\
Symmetric Passive & 2 & 1 \\
Client & 3 & 4 \\
Server & 4 & 3 \\
Broadcast Server & 5 & 5 \\
Broadcast Client & 6 & N/A \\
\hline
\end{tabular}
\label{association_and_packet_modes}
\caption{Association and Packet Modes}
\end{table}

                +-----------------------------------+
                | Packet Variable <--   Variable    |
                +-----------------------------------+
                | x.leap        <--     s.leap      |
                | x.version     <--     r.version   |
                | x.mode        <--     4           |
                | x.stratum     <--     s.stratum   |
                | x.poll        <--     r.poll      |
                | x.precision   <--     s.precision |
                | x.rootdelay   <--     s.rootdelay |
                | x.rootdisp    <--     s.rootdisp  |
                | x.refid       <--     s.refid     |
                | x.reftime     <--     s.reftime   |
                | x.org         <--     r.xmt       |
                | x.rec         <--     r.dst       |
                | x.xmt         <--     clock       |
                | x.keyid       <--     r.keyid     |
                | x.digest      <--     md5 digest  |
                +-----------------------------------+

                Figure 31: fast\_xmit Packet Header

An SNTP client implementing the on-wire protocol has a single server
and no dependent clients.  It can operate with any subset of the NTP
on-wire protocol, the simplest approach using only the transmit
timestamp of the server packet and ignoring all other fields.
However, the additional complexity to implement the full on-wire
protocol is minimal so that a full implementation is encouraged.
