\chapter{Definitions}

A number of technical terms are defined in this section. A timescale
is a frame of reference where time is expressed as the value of a
monotonically increasing binary counter with an indefinite number of
bits. It counts in seconds and fractions of a second, when a decimal
point is employed. The Coordinated Universal Time (UTC) timescale is
defined by ITU-R TF.460 [ITU-R_TF.460]. Under the auspices of the
Metre Convention of 1865, in 1975 the CGPM [CGPM] strongly endorsed
the use of UTC as the basis for civil time.

The Coordinated Universal Time (UTC) timescale represents mean solar
time as disseminated by national standards laboratories. The system
time is represented by the system clock maintained by the hardware
and operating system. The goal of the NTP algorithms is to minimize
both the time difference and frequency difference between UTC and the
system clock. When these differences have been reduced below nominal
tolerances, the system clock is said to be synchronized to UTC.

The date of an event is the UTC time at which the event takes place.
Dates are ephemeral values designated with uppercase T. Running time
is another timescale that is coincident to the synchronization
function of the NTP program.

A timestamp T(t) represents either the UTC date or time offset from
UTC at running time t. Which meaning is intended should be clear
from the context. Let T(t) be the time offset, R(t) the frequency
offset, and D(t) the aging rate (first derivative of R(t) with
respect to t). Then, if T(t_0) is the UTC time offset determined at
t = t_0, the UTC time offset at time t is

T(t) = T(t_0) + R(t_0)(t-t_0) + 1/2 * D(t_0)(t-t_0)^2 + e,

where e is a stochastic error term discussed later in this document.
While the D(t) term is important when characterizing precision
oscillators, it is ordinarily neglected for computer oscillators. In
this document, all time values are in seconds (s) and all frequency
values are in seconds-per-second (s/s). It is sometimes convenient
to express frequency offsets in parts-per-million (ppm), where 1 ppm
is equal to 10^(-6) s/s.

It is important in computer timekeeping applications to assess the
performance of the timekeeping function. The NTP performance model
includes four statistics that are updated each time a client makes a
measurement with a server. The offset (theta) represents the
maximum-likelihood time offset of the server clock relative to the
system clock. The delay (delta) represents the round-trip delay
between the client and server. The dispersion (epsilon) represents
the maximum error inherent in the measurement. It increases at a
rate equal to the maximum disciplined system clock frequency
tolerance (PHI), typically 15 ppm. The jitter (psi) is defined as
the root-mean-square (RMS) average of the most recent offset
differences, and it represents the nominal error in estimating the
offset.

While the theta, delta, epsilon, and psi statistics represent
measurements of the system clock relative to each server clock
separately, the NTP protocol includes mechanisms to combine the
statistics of several servers to more accurately discipline and
calibrate the system clock. The system offset (THETA) represents the
maximum-likelihood offset estimate for the server population. The
system jitter (PSI) represents the nominal error in estimating the
system offset. The delta and epsilon statistics are accumulated at
each stratum level from the reference clock to produce the root delay
(DELTA) and root dispersion (EPSILON) statistics. The
synchronization distance (LAMBDA) equal to EPSILON + DELTA / 2
represents the maximum error due to all causes. The detailed
formulations of these statistics are given in Section 11.2. They are
available to the dependent applications in order to assess the
performance of the synchronization function.
