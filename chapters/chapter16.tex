\chapter{IANA Considerations}
\label{section-16}

UDP/TCP Port 123 was previously assigned by IANA for this protocol.
The IANA has assigned the IPv4 multicast group address 224.0.1.1 and
the IPv6 multicast address ending :101 for NTP. This document
introduces NTP extension fields allowing for the development of
future extensions to the protocol, where a particular extension is to
be identified by the Field Type sub-field within the extension field.
IANA has established and will maintain a registry for Extension Field
Types associated with this protocol, populating this registry with no
initial entries. As future needs arise, new Extension Field Types
may be defined. Following the policies outlined in \cite{RFC5226}, new
values are to be defined by IETF Review.

The IANA has created a new registry for NTP Reference Identifier
codes. This includes the current codes defined in Section~\ref{section-7-3}, and
may be extended on a First-Come-First-Serve (FCFS) basis. The format
of the registry is:

\begin{table}[htb]
  \center
  \begin{tabular}{| l | l |}
    \hline
    ID   & Clock Source \\
    \hline
    \hline
    GOES & Geosynchronous Orbit Environment Satellite \\
    GPS  & Global Position System                     \\
    ... & ...                                       \\
    \hline
  \end{tabular}
  \caption{Reference Identifier Codes}
  \label{reference_identifier_codes}
\end{table}

The IANA has created a new registry for NTP Kiss-o'-Death codes.
This includes the current codes defined in Section~\ref{section-7-4}, and may be
extended on a FCFS basis. The format of the registry is:

\begin{table}[htb]
  \center
  \begin{tabular}{| l | l |}
    \hline
    Code & Meaning \\
    \hline
    \hline
    ACST & The association belongs to a unicast server. \\
    AUTH & Server authentication failed. \\
    ... & ... \\
    \hline
  \end{tabular}
  \caption{Kiss Codes}
  \label{kiss_codes}
\end{table}

For both Reference Identifiers and Kiss-o'-Death codes, IANA is
requested to never assign a code beginning with the character ``X'', as
this is reserved for experimentation and development.
