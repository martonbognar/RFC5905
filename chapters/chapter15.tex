.  Security Considerations

   NTP security requirements are even more stringent than most other
   distributed services.  First, the operation of the authentication
   mechanism and the time synchronization mechanism are inextricably
   intertwined.  Reliable time synchronization requires cryptographic
   keys that are valid only over a designated time interval; but, time
   intervals can be enforced only when participating servers and clients
      are reliably synchronized to UTC.  In addition, the NTP subnet is
   hierarchical by nature, so time and trust flow from the primary
   servers at the root through secondary servers to the clients at the
   leaves.

   An NTP client can claim to have authentic time to dependent
   applications only if all servers on the path to the primary servers
   are authenticated.  In NTP each server authenticates the next lower
   stratum servers and authenticates by induction the lowest stratum
   (primary) servers.  It is important to note that authentication in
   the context of NTP does not necessarily imply the time is correct.
   An NTP client mobilizes a number of concurrent associations with
   different servers and uses a crafted agreement algorithm to pluck
   truechimers from the population possibly including falsetickers.

   The NTP specification assumes that the goal of the intruder is to
   inject false time values, disrupt the protocol, or clog the network,
   servers, or clients with spurious packets that exhaust resources and
   deny service to legitimate applications.  There are a number of
   defense mechanisms already built in the NTP architecture, protocol,
   and algorithms.  The on-wire timestamp exchange scheme is inherently
   resistant to spoofing, packet-loss, and replay attacks.  The
   engineered clock filter, selection and clustering algorithms are
   designed to defend against evil cliques of Byzantine traitors.  While
   not necessarily designed to defeat determined intruders, these
   algorithms and accompanying sanity checks have functioned well over
   the years to deflect improperly operating but presumably friendly
   scenarios.  However, these mechanisms do not securely identify and
   authenticate servers to clients.  Without specific further
   protection, an intruder can inject any or all of the following
   attacks:

   1.  An intruder can intercept and archive packets forever, as well as
       all the public values ever generated and transmitted over the
       net.

   2.  An intruder can generate packets faster than the server, network
       or client can process them, especially if they require expensive
       cryptographic computations.

   3.  In a wiretap attack, the intruder can intercept, modify, and
       replay a packet.  However, it cannot permanently prevent onward
       transmission of the original packet; that is, it cannot break the
       wire, only tell lies and congest it.  Generally, the modified
       packet cannot arrive at the victim before the original packet,
       nor does it have the server private keys or identity parameters.

   4.  In a middleman or masquerade attack, the intruder is positioned
       between the server and client, so it can intercept, modify and
       replay a packet and prevent onward transmission of the original
       packet.  However, the middleman does not have the server private
       keys.

   The NTP security model assumes the following possible limitations:

   1.  The running times for public key algorithms are relatively long
       and highly variable.  In general, the performance of the time
       synchronization function is badly degraded if these algorithms
       must be used for every NTP packet.

   2.  In some modes of operation, it is not feasible for a server to
       retain state variables for every client.  It is however feasible
       to regenerated them for a client upon arrival of a packet from
       that client.

   3.  The lifetime of cryptographic values must be enforced, which
       requires a reliable system clock.  However, the sources that
       synchronize the system clock must be trusted.  This circular
       interdependence of the timekeeping and authentication functions
       requires special handling.

   4.  Client security functions must involve only public values
       transmitted over the net.  Private values must never be disclosed
       beyond the machine on which they were created, except in the case
       of a special trusted agent (TA) assigned for this purpose.

   Unlike the Secure Shell (SSH) security model, where the client must
   be securely authenticated to the server, in NTP the server must be
   securely authenticated to the client.  In SSH, each different
   interface address can be bound to a different name, as returned by a
   reverse-DNS query.  In this design, separate public/private key pairs
   may be required for each interface address with a distinct name.  A
   perceived advantage of this design is that the security compartment
   can be different for each interface.  This allows a firewall, for
   instance, to require some interfaces to authenticate the client and
   others not.

   In the case of NTP as specified herein, NTP broadcast clients are
   vulnerable to disruption by misbehaving or hostile SNTP or NTP
   broadcast servers elsewhere in the Internet.  Such disruption can be
   minimized by several approaches.  Filtering can be employed to limit
   the access of NTP clients to known or trusted NTP broadcast servers.
   Such filtering will prevent malicious traffic from reaching the NTP
   clients.  Cryptographic authentication at the client will only allow

   timing information from properly signed NTP messages to be utilized
   in synchronizing its clock.  Higher levels of authentication may be
   gained by the use of the Autokey mechanism [RFC5906].

   Section 8 describes a potential security concern with the replay of
   client requests.  Following the recommendations in that section
   provides protection against such attacks.

   It should be noted that this specification is describing an existing
   implementation.  While the security shortfalls of the MD5 algorithm
   are well-known, its use in the NTP specification is consistent with
   widescale deployment in the Internet community.
