\chapter{Protocol Modes}
\label{section-3}

There are three NTP protocol variants: symmetric, client/server, and
broadcast. Each is associated with an association mode (a
description of the relationship between two NTP speakers) as shown in
Table~\ref{association_and_packet_modes}. In addition, persistent associations are mobilized upon
startup and are never demobilized. Ephemeral associations are
mobilized upon the arrival of a packet and are demobilized upon error
or timeout.

\begin{table}[htb]
\center
\begin{tabular}{c | c | c}
Association Mode & Assoc. Mode Value & Packet Mode Value \\
\hline
\hline
Symmetric Active & 1 & 1 or 2 \\
Symmetric Passive & 2 & 1 \\
Client & 3 & 4 \\
Server & 4 & 3 \\
Broadcast Server & 5 & 5 \\
Broadcast Client & 6 & N/A \\
\hline
\end{tabular}
\label{association_and_packet_modes}
\caption{Association and Packet Modes}
\end{table}

In the client/server variant, a persistent client sends packet mode 4
packets to a server, which returns packet mode 3 packets. Servers
provide synchronization to one or more clients, but do not accept
synchronization from them. A server can also be a reference clock
driver that obtains time directly from a standard source such as a
GPS receiver or telephone modem service. In this variant, clients
pull synchronization from servers.

In the symmetric variant, a peer operates as both a server and client
using either a symmetric active or symmetric passive association. A
persistent symmetric active association sends symmetric active (mode
1) packets to a symmetric active peer association. Alternatively, an
ephemeral symmetric passive association can be mobilized upon the
arrival of a symmetric active packet with no matching association.
That association sends symmetric passive (mode 2) packets and
persists until error or timeout. Peers both push and pull
synchronization to and from each other. For the purposes of this
document, a peer operates like a client, so references to client
imply peer as well.

In the broadcast variant, a persistent broadcast server association
sends periodic broadcast server (mode 5) packets that can be received
by multiple clients. Upon reception of a broadcast server packet
without a matching association, an ephemeral broadcast client (mode
6) association is mobilized and persists until error or timeout. It
is useful to provide an initial volley where the client operating in
client mode exchanges several packets with the server, so as to
calibrate the propagation delay and to run the Autokey security
protocol, after which the client reverts to broadcast client mode. A
broadcast server pushes synchronization to clients and other servers.

Loosely following the conventions established by the telephone
industry, the level of each server in the hierarchy is defined by a
stratum number. Primary servers are assigned stratum one; secondary
servers at each lower level are assigned stratum numbers one greater
than the preceding level. As the stratum number increases, its
accuracy degrades depending on the particular network path and system
clock stability. Mean errors, measured by synchronization distances,
increase approximately in proportion to stratum numbers and measured
round-trip delay.

As a standard practice, timing network topology should be organized
to avoid timing loops and minimize the synchronization distance. In
NTP, the subnet topology is determined using a variant of the
Bellman-Ford distributed routing algorithm, which computes the
shortest-path spanning tree rooted on the primary servers. As a
result of this design, the algorithm automatically reorganizes the
subnet, so as to produce the most accurate and reliable time, even
when there are failures in the timing network.

\section{Dynamic Server Discovery}
\label{section-3-1}

There are two special associations, manycast client and manycast
server, which provide a dynamic server discovery function. There are
two types of manycast client associations: persistent and ephemeral.
The persistent manycast client sends client (mode 3) packets to a
designated IPv4 or IPv6 broadcast or multicast group address.
Designated manycast servers within range of the time-to-live (TTL)
field in the packet header listen for packets with that address. If
a server is suitable for synchronization, it returns an ordinary
server (mode 4) packet using the client's unicast address. Upon
receiving this packet, the client mobilizes an ephemeral client (mode
3) association. The ephemeral client association persists until
error or timeout.

A manycast client continues sending packets to search for a minimum
number of associations. It starts with a TTL equal to one and
continuously adding one to it until the minimum number of
associations is made or when the TTL reaches a maximum value. If the
TTL reaches its maximum value and yet not enough associations are
mobilized, the client stops transmission for a time-out period to
clear all associations, and then repeats the search cycle. If a
minimum number of associations has been mobilized, then the client
starts transmitting one packet per time-out period to maintain the
associations. Field constraints limit the minimum value to 1 and the
maximum to 255. These limits may be tuned for individual application
needs.

The ephemeral associations compete among themselves. As new
ephemeral associations are mobilized, the client runs the mitigation
algorithms described in Sections~\ref{section-10}~and~\ref{section-11-2} for the best candidates
out of the population, the remaining ephemeral associations are timed
out and demobilized. In this way, the population includes only the
best candidates that have most recently responded with an NTP packet
to discipline the system clock.
