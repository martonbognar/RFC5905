 Implementation Model
 Figure 2 shows the architecture of a typical, multi-threaded
 implementation. It includes two processes dedicated to each server,
 a peer process to receive messages from the server or reference
 clock, and a poll process to transmit messages to the server or
 reference clock.
 .....................................................................
 . Remote . Peer/Poll . System . Clock .
 . Servers . Processes . Process .Discipline.
 . . . . Process .
 .+--------+. +-----------+. +------------+ . .
 .| |->| |. | | . .
 .|Server 1| |Peer/Poll 1|->| | . .
 .| |<-| |. | | . .
 .+--------+. +-----------+. | | . .
 . . ^ . | | . .
 . . | . | | . .
 .+--------+. +-----------+. | | +-----------+. .
 .| |->| |. | Selection |->| |. +------+ .
 .|Server 2| |Peer/Poll 2|->| and | | Combine |->| Loop | .
 .| |<-| |. | Cluster | | Algorithm |. |Filter| .
 .+--------+. +-----------+. | Algorithms |->| |. +------+ .
 . . ^ . | | +-----------+. | .
 . . | . | | . | .
 .+--------+. +-----------+. | | . | .
 .| |->| |. | | . | .
 .|Server 3| |Peer/Poll 3|->| | . | .
 .| |<-| |. | | . | .
 .+--------+. +-----------+. +------------+ . | .
 ....................^.........................................|......
 | . V .
 | . +-----+ .
 +--------------------------------------| VFO | .
 . +-----+ .
 . Clock .
 . Adjust .
 . Process .
 ............
 Figure 2: Implementation Model
These processes operate on a common data structure, called an
 association, which contains the statistics described above along with
 various other data described in Section 9. A client sends packets to
 one or more servers and then processes returned packets when they are
 received. The server interchanges source and destination addresses
 and ports, overwrites certain fields in the packet and returns it
 immediately (in the client/server mode) or at some time later (in the
 symmetric modes). As each NTP message is received, the offset theta
 between the peer clock and the system clock is computed along with
 the associated statistics delta, epsilon, and psi.
 The system process includes the selection, cluster, and combine
 algorithms that mitigate among the various servers and reference
 clocks to determine the most accurate and reliable candidates to
 synchronize the system clock. The selection algorithm uses Byzantine
 fault detection principles to discard the presumably incorrect
 candidates called "falsetickers" from the incident population,
 leaving only good candidates called "truechimers". A truechimer is a
 clock that maintains timekeeping accuracy to a previously published
 and trusted standard, while a falseticker is a clock that shows
 misleading or inconsistent time. The cluster algorithm uses
 statistical principles to find the most accurate set of truechimers.
 The combine algorithm computes the final clock offset by
 statistically averaging the surviving truechimers.
 The clock discipline process is a system process that controls the
 time and frequency of the system clock, here represented as a
 variable frequency oscillator (VFO). Timestamps struck from the VFO
 close the feedback loop that maintains the system clock time.
 Associated with the clock discipline process is the clock-adjust
 process, which runs once each second to inject a computed time offset
 and maintain constant frequency. The RMS average of past time offset
 differences represents the nominal error or system clock jitter. The
 RMS average of past frequency offset differences represents the
 oscillator frequency stability or frequency wander. These terms are
 given precise interpretation in Section 11.3.
 A client sends messages to each server with a poll interval of 2^tau
 seconds, as determined by the poll exponent tau. In NTPv4, tau
 ranges from 4 (16 s) to 17 (36 h). The value of tau is determined by
 the clock discipline algorithm to match the loop-time constant T_c =
 2^tau. In client/server mode, the server responds immediately;
 however, in symmetric modes, each of two peers manages tau as a
 function of current system offset and system jitter, so they may not
 agree with the same value. It is important that the dynamic behavior
 of the clock discipline algorithm be carefully controlled in order to
 maintain stability in the NTP subnet at large. This requires that
the peers agree on a common tau equal to the minimum poll exponent of
 both peers. The NTP protocol includes provisions to properly
 negotiate this value.
 The implementation model includes some means to set and adjust the
 system clock. The operating system is assumed to provide two
 functions: one to set the time directly, for example, the Unix
 settimeofday() function, and another to adjust the time in small
 increments advancing or retarding the time by a designated amount,
 for example, the Unix adjtime() function. In this and following
 references, parentheses following a name indicate reference to a
 function rather than a simple variable. In the intended design the
 clock discipline process uses the adjtime() function if the
 adjustment is less than a designated threshold, and the
 settimeofday() function if above the threshold. The manner in which
 this is done and the value of the threshold as described in
 Section 10.
