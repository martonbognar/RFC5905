 Clock Filter Algorithm

   The clock filter algorithm is part of the peer process.  It grooms
   the stream of on-wire data to select the samples most likely to
   represent accurate time.  The algorithm produces the variables shown
   in Figure 19, including the offset (theta), delay (delta), dispersion
   (epsilon), jitter (psi), and time of arrival (t).  These data are
   used by the mitigation algorithms to determine the best and final
   offset used to discipline the system clock.  They are also used to
   determine the server health and whether it is suitable for
   synchronization.

   The clock filter algorithm saves the most recent sample tuples
   (theta, delta, epsilon, t) in the filter structure, which functions
   as an 8-stage shift register.  The tuples are saved in the order that
   packets arrive.  Here, t is the packet time of arrival according to
   the seconds counter and should not be confused with the peer variable
   tp.

   The following scheme is used to ensure sufficient samples are in the
   filter and that old stale data are discarded.  Initially, the tuples
   of all stages are set to the dummy tuple (0, MAXDISP, MAXDISP, 0).
   As valid packets arrive, tuples are shifted into the filter causing
   old tuples to be discarded, so eventually only valid tuples remain.
     If the three low-order bits of the reach register are zero,
   indicating three poll intervals have expired with no valid packets
   received, the poll process calls the clock filter algorithm with a
   dummy tuple just as if the tuple had arrived from the network.  If
   this persists for eight poll intervals, the register returns to the
   initial condition.

   In the next step, the shift register stages are copied to a temporary
   list and the list sorted by increasing delta.  Let i index the stages
   starting with the lowest delta.  If the first tuple epoch t_0 is not
   later than the last valid sample epoch tp, the routine exits without
   affecting the current peer variables.  Otherwise, let epsilon_i be
   the dispersion of the ith entry, then

                     i=n-1
                     ---     epsilon_i
      epsilon =       \     ----------
                      /        (i+1)
                     ---     2
                     i=0

   is the peer dispersion p.disp.  Note the overload of epsilon, whether
   input to the clock filter or output, the meaning should be clear from
   context.

   The observer should note (a) if all stages contain the dummy tuple
   with dispersion MAXDISP, the computed dispersion is a little less
   than 16 s, (b) each time a valid tuple is shifted into the register,
   the dispersion drops by a little less than half, depending on the
   valid tuples dispersion, and (c) after the fourth valid packet the
   dispersion is usually a little less than 1 s, which is the assumed
   value of the MAXDIST parameter used by the selection algorithm to
   determine whether or not the peer variables are acceptable.

   Let the first stage offset in the sorted list be theta_0; then, for
   the other stages in any order, the jitter is the RMS average

                          +-----                 -----+^1/2
                          |  n-1                      |
                          |  ---                      |
                  1       |  \                     2  |
      psi   =  -------- * |  /    (theta_0-theta_j)   |
                (n-1)     |  ---                      |
                          |  j=1                      |
                          +-----                 -----+

   where n is the number of valid tuples in the filter (n > 1).  In
   order to ensure consistency and avoid divide exceptions in other
 computations, the psi is bounded from below by the system precision
   s.rho expressed in seconds.  While not in general considered a major
   factor in ranking server quality, jitter is a valuable indicator of
   fundamental timekeeping performance and network congestion state.  Of
   particular importance to the mitigation algorithms is the peer
   synchronization distance, which is computed from the delay and
   dispersion.

   lambda = (delta / 2) + epsilon.

   Note that epsilon and therefore lambda increase at rate PHI.  The
   lambda is not a state variable, since lambda is recalculated at each
   use.  It is a component of the root synchronization distance used by
   the mitigation algorithms as a metric to evaluate the quality of time
   available from each server.

   It is important to note that, unlike NTPv3, NTPv4 associations do not
   show a timeout condition by setting the stratum to 16 and leap
   indicator to 3.  The association variables retain the values
   determined upon arrival of the last packet.  In NTPv4, lambda
   increases with time, so eventually the synchronization distance
   exceeds the distance threshold MAXDIST, in which case the association
   is considered unfit for synchronization.

   An example implementation of the clock filter algorithm is shown in
   the clock_filter() routine of Appendix A.5.2.
