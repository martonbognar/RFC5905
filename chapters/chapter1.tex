Introduction
 This document defines the Network Time Protocol version 4 (NTPv4),
 which is widely used to synchronize system clocks among a set of
 distributed time servers and clients. It describes the core
 architecture, protocol, state machines, data structures, and
 algorithms. NTPv4 introduces new functionality to NTPv3, as
 described in [RFC1305], and functionality expanded from Simple NTP
 version 4 (SNTPv4) as described in [RFC4330] (SNTPv4 is a subset of
 NTPv4). This document obsoletes [RFC1305] and [RFC4330]. While
 certain minor changes have been made in some protocol header fields,
 these do not affect the interoperability between NTPv4 and previous
 versions of NTP and SNTP.
 The NTP subnet model includes a number of widely accessible primary
 time servers synchronized by wire or radio to national standards.
 The purpose of the NTP protocol is to convey timekeeping information
 from these primary servers to secondary time servers and clients via
 both private networks and the public Internet. Precisely tuned
 algorithms mitigate errors that may result from network disruptions,
 server failures, and possible hostile actions. Servers and clients
 are configured such that values flow towards clients from the primary
 servers at the root via branching secondary servers.
 The NTPv4 design overcomes significant shortcomings in the NTPv3
 design, corrects certain bugs, and incorporates new features. In
 particular, expanded NTP timestamp definitions encourage the use of
 the floating double data type throughout the implementation. As a
 result, the time resolution is better than one nanosecond, and
 frequency resolution is less than one nanosecond per second.
 Additional improvements include a new clock discipline algorithm that
 is more responsive to system clock hardware frequency fluctuations.
 Typical primary servers using modern machines are precise within a
 few tens of microseconds. Typical secondary servers and clients on
 fast LANs are within a few hundred microseconds with poll intervals
 up to 1024 seconds, which was the maximum with NTPv3. With NTPv4,
 servers and clients are precise within a few tens of milliseconds
 with poll intervals up to 36 hours.
 The main body of this document describes the core protocol and data
 structures necessary to interoperate between conforming
 implementations. Appendix A contains a full-featured example in the
 form of a skeleton program, including data structures and code
 segments for the core algorithms as well as the mitigation algorithms
 used to enhance reliability and accuracy. While the skeleton program
 and other descriptions in this document apply to a particular
 implementation, they are not intended as the only way the required
 functions can be implemented. The contents of Appendix A are non-normative examples designed to illustrate the protocol’s operation
 and are not a requirement for a conforming implementation. While the
 NTPv3 symmetric key authentication scheme described in this document
 has been carried over from NTPv3, the Autokey public key
 authentication scheme new to NTPv4 is described in [RFC5906].
 The NTP protocol includes modes of operation described in Section 2
 using data types described in Section 6 and data structures described
 in Section 7. The implementation model described in Section 5 is
 based on a threaded, multi-process architecture, although other
 architectures could be used as well. The on-wire protocol described
 in Section 8 is based on a returnable-time design that depends only
 on measured clock offsets, but does not require reliable message
 delivery. Reliable message delivery such as TCP [RFC0793] can
 actually make the delivered NTP packet less reliable since retries
 would increase the delay value and other errors. The synchronization
 subnet is a self-organizing, hierarchical, master-slave network with
 synchronization paths determined by a shortest-path spanning tree and
 defined metric. While multiple masters (primary servers) may exist,
 there is no requirement for an election protocol.
 This document includes material from [ref9], which contains flow
 charts and equations unsuited for RFC format. There is much
 additional information in [ref7], including an extensive technical
 analysis and performance assessment of the protocol and algorithms in
 this document. The reference implementation is available at
 www.ntp.org.
 The remainder of this document contains numerous variables and
 mathematical expressions. Some variables take the form of Greek
 characters, which are spelled out by their full case-sensitive name.
 For example, DELTA refers to the uppercase Greek character, while
 delta refers to the lowercase character. Furthermore, subscripts are
 denoted with ’_’; for example, theta_i refers to the lowercase Greek
 character theta with subscript i, or phonetically theta sub i. In
 this document, all time values are in seconds (s), and all
 frequencies will be specified as fractional frequency offsets (FFOs)
 (pure number). It is often convenient to express these FFOs in parts
 per million (ppm).

1.1. Requirements Notation
 The key words "MUST", "MUST NOT", "REQUIRED", "SHALL", "SHALL NOT",
 "SHOULD", "SHOULD NOT", "RECOMMENDED", "MAY", and "OPTIONAL" in this
 document are to be interpreted as described in [RFC2119].
