\chapter{Clock-Adjust Process}

The actual clock-adjust process runs at one-second intervals to add
the frequency correction and a fixed percentage of the residual
offset theta\_r.  The theta\_r is, in effect, the exponential decay of
the theta value produced by the loop filter at each update.  The TC
parameter scales the time constant to match the poll interval for
convenience.  Note that the dispersion EPSILON increases by PHI at
each second.

The clock-adjust process includes a timer interrupt facility driving
the seconds counter c.t.  It begins at zero when the service starts
and increments once each second.  At each interrupt, the
clock\_adjust() routine is called to incorporate the clock discipline
time and frequency adjustments, then the associations are scanned to
determine if the seconds counter equals or exceeds the p.next state
variable defined in the next section.  If so, the poll process is
called to send a packet and compute the next p.next value.

An example of the clock-adjust process is shown by the clock\_adjust()
routine in Appendix A.5.6.1.
