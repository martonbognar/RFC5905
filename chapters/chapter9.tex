\chapter{Peer Process}

The process descriptions to follow include a listing of the important
state variables followed by an overview of the process operations
implemented as routines. Frequent reference is made to the skeleton
in the appendix. The skeleton includes C-language fragments that
describe the functions in more detail. It includes the parameters,
variables, and declarations necessary for a conforming NTPv4
implementation. However, many additional variables and routines may
be necessary in a working implementation.

The peer process is called upon arrival of a server or peer packet.
It runs the on-wire protocol to determine the clock offset and round-
trip delay and computes statistics used by the system and poll
processes. Peer variables are instantiated in the association data
structure when the structure is initialized and updated by arriving
packets. There is a peer process, poll process, and association
process for each server.

\section{Peer Process Variables}

Figures 16, 17, 18, and 19 summarize the common names, formula names,
and a short description of the peer variables. The common names and
formula names are interchangeable; formula names are intended to
increase readability of equations in this specification. Unless
noted otherwise, all peer variables have assumed prefix p.

\begin{table}[htb]
\center
\begin{tabular}{c | c | c}
Name & Formula & Description \\
\hline
\hline
srcaddr & srcaddr & source address \\
srcport & srcport & source port \\
dstaddr & dstaddr & destination address \\
dstport & destport & destination port \\
keyid & keyid & key identifier key ID \\
\hline
\end{tabular}
\label{peer_process_configuration_variables}
\caption{Peer Process Configuration Variables}
\end{table}

\begin{table}[htb]
\center
\begin{tabular}{c | c | c}
Name & Formula & Description \\
\hline
\hline
leap & leap & leap indicator \\
version & version & version number \\
mode & mode & mode \\
stratum & stratum & stratum \\
ppoll & ppoll & peer poll exponent \\
rootdelay & delta\_r & root delay \\
rootdisp & epsilon\_r & root dispersion \\
refid & refid & reference ID \\
reftime & reftime & reference timestamp \\
\hline
\end{tabular}
\label{peer_process_packet_variables}
\caption{Peer Process Packet Variables}
\end{table}

\begin{table}[htb]
\center
\begin{tabular}{c | c | c}
Name & Formula & Description \\
\hline
\hline
org & T1 & origin timestamp \\
rec & T2 & receive timestamp \\
xmt & T3 & transmit timestamp \\
t & t & packet time \\
\hline
\end{tabular}
\label{peer_process_timestamp_variables}
\caption{Peer Process Timestamp Variables}
\end{table}

\begin{table}[htb]
\center
\begin{tabular}{c | c | c}
Name & Formula & Description \\
\hline
\hline
offset & theta & clock offset \\
delay & delta & round-trip delay\\
disp & epsilon & dispersion \\
jitter & psi & jitter \\
filter & filter & clock filter \\
tp & t\_p & filter time \\
\hline
\end{tabular}
\label{peer_process_statistics_variables}
\caption{Peer Process Statistics Variables}
\end{table}

The following configuration variables are normally initialized when
the association is mobilized, either from a configuration file or
upon the arrival of the first packet for an unknown association.

srcaddr: IP address of the remote server or reference clock. This
becomes the destination IP address in packets sent from this
association.

srcport: UDP port number of the server or reference clock. This
becomes the destination port number in packets sent from this
association. When operating in symmetric modes (1 and 2), this field
must contain the NTP port number PORT (123) assigned by the IANA. In
other modes, it can contain any number consistent with local policy.

dstaddr: IP address of the client. This becomes the source IP
address in packets sent from this association.

dstport: UDP port number of the client, ordinarily the NTP port
number PORT (123) assigned by the IANA. This becomes the source port
number in packets sent from this association.

keyid: Symmetric key ID for the 128-bit MD5 key used to generate and
verify the MAC. The client and server or peer can use different
values, but they must map to the same key.

The variables defined in Figure 17 are updated from the packet header
as each packet arrives. They are interpreted in the same way as the
packet variables of the same names. It is convenient for later
processing to convert the NTP short format packet values r.rootdelay
and r.rootdisp to floating doubles as peer variables.

The variables defined in Figure 18 include the timestamps exchanged
by the on-wire protocol in Section 8. The t variable is the seconds
counter c.t associated with these values. The c.t variable is
maintained by the clock-adjust process described in Section 12. It
counts the seconds since the service was started. The variables
defined in Figure 19 include the statistics computed by the
clock\_filter() routine described in Section 10. The tp variable is
the seconds counter associated with these values.

\section{Peer Process Operations}

The receive routine defines the process flow upon the arrival of a
packet. An example is described by the receive() routine in
Appendix A.5.1. There is no specific method required for access
control, although it is recommended that implementations include such
a scheme, which is similar to many others now in widespread use. The
access() routine in Appendix A.5.4 describes a method of implementing
access restrictions using an access control list (ACL). Format
checks require correct field length and alignment, acceptable version
number (1-4), and correct extension field syntax, if present.

There is no specific requirement for authentication; however, if
authentication is implemented, then the MD5-keyed hash algorithm
described in [RFC1321] must be supported.

Next, the association table is searched for matching source address
and source port, for example, using the find\_assoc() routine in
Appendix A.5.1. Figure 20 is a dispatch table where the columns
correspond to the packet mode and rows correspond to the association
mode. The intersection of the association and packet modes
dispatches processing to one of the following steps.

\begin{table}[htb]
\center
\begin{tabular}{c | c | c | c | c | c}
 & \multicolumn{5}{c}{Packet Mode} \\
Association Mode & 1 & 2 & 3 & 4 & 5 \\
\hline
\hline
No Association 0 & NEWPS & DSCRD & FXMIT & MANY & NEWBC \\
Symm. Active 1 & PROC & PROC & DSCRD & DSCRD & DSCRD \\
Symm. Passive 2 & PROC & ERR & DSCRD & DSCRD & DSCRD \\
Client 3 & DSCRD & DSCRD & DSCRD & PROC & DSCRD \\
Server 4 & DSCRD & DSCRD & DSCRD & DSCRD & DSCRD \\
Broadcast 5 & DSCRD & DSCRD & DSCRD & DSCRD & DSCRD \\
Bcast Client 6 & DSCRD & DSCRD & DSCRD & DSCRD & PROC \\
\hline
\end{tabular}
\label{peer_dispatch_table}
\caption{Peer Dispatch Table}
\end{table}

DSCRD. This indicates a non-fatal violation of protocol as the
result of a programming error, long-delayed packet, or replayed
packet. The peer process discards the packet and exits.

ERR. This indicates a fatal violation of protocol as the result of a
programming error, long-delayed packet, or replayed packet. The peer
process discards the packet, demobilizes the symmetric passive
association, and exits.

FXMIT. This indicates a client (mode 3) packet matching no
association (mode 0). If the destination address is not a broadcast
address, the server constructs a server (mode 4) packet and returns
it to the client without retaining state. The server packet header
is constructed. An example is described by the fast\_xmit() routine
in Appendix A.5.3. The packet header is assembled from the receive
packet and system variables as shown in Figure 21. If the
s.rootdelay and s.rootdisp system variables are stored in floating
double, they must be converted to NTP short format first.

\begin{table}[htb]
\center
\begin{tabular}{c | c | c}
Packet Variable & --> & Variable \\
\hline
\hline
r.leap & --> & p.leap \\
r.mode & --> & p.mode \\
r.stratum & --> & p.stratum \\
r.poll & --> & p.ppoll \\
r.rootdelay & --> & p.rootdelay \\
r.rootdisp & --> & p.rootdisp \\
r.refid & --> & p.refid \\
r.reftime & --> & p.reftime \\
r.keyid & --> & p.keyid \\
\hline
\end{tabular}
\label{receive_packet_header}
\caption{Receive Packet Header}
\end{table}

Note that, if authentication fails, the server returns a special
message called a crypto-NAK. This message includes the normal NTP
header data shown in Figure 8, but with a MAC consisting of four
octets of zeros. The client MAY accept or reject the data in the
message. After these actions, the peer process exits.

If the destination address is a multicast address, the sender is
operating in manycast client mode. If the packet is valid and the
server stratum is less than the client stratum, the server sends an
ordinary server (mode 4) packet, but one which uses its unicast
destination address. A crypto-NAK is not sent if authentication
fails. After these actions, the peer process exits.

MANY: This indicates a server (mode 4) packet matching no
association. Ordinarily, this can happen only as the result of a
manycast server reply to a previously sent multicast client packet.

If the packet is valid, an ordinary client (mode 3) association is
mobilized and operation continues as if the association was mobilized
by the configuration file.

NEWBC. This indicates a broadcast (mode 5) packet matching no
association. The client mobilizes either a client (mode 3) or
broadcast client (mode 6) association. Examples are shown in the
mobilize() and clear() routines in Appendix A.2. Then, the packet is
validated and the peer variables initialized. An example is provided
by the packet() routine in Appendix A.5.1.1.

If the implementation supports no additional security or calibration
functions, the association mode is set to broadcast client (mode 6)
and the peer process exits. Implementations supporting public key
authentication MAY run the Autokey or equivalent security protocol.
Implementations SHOULD set the association mode to 3 and run a short
client/server exchange to determine the propagation delay. Following
the exchange, the association mode is set to 6 and the peer process
continues in listen-only mode. Note the distinction between a mode-6
packet, which is reserved for the NTP monitor and control functions,
and a mode-6 association.

NEWPS. This indicates a symmetric active (mode 1) packet matching no
association. The client mobilizes a symmetric passive (mode 2)
association. An example is shown in the mobilize() and clear()
routines in Appendix A.2. Processing continues in the PROC section
below.

PROC. This indicates a packet matching an existing association. The
packet timestamps are carefully checked to avoid invalid, duplicate,
or bogus packets. Additional checks are summarized in Figure 22.
Note that all packets, including a crypto-NAK, are considered valid
only if they survive these tests.

\begin{table}[htb]
\center
\begin{tabular}{c | c | c}
Packet Type & Description \\
\hline
\hline
1 duplicate packet & The packet is at best an old duplicate or at worst a replay by a hacker. This can happen in symmetric modes if the poll intervals are uneven. \\
2 bogus packet & \\
3 invalid & One or more timestamp fields are invalid. This normally happens in symmetric modes when one peer sends the first packet to the other and before the other has received its first reply. \\
4 access denied & The access controls have blacklisted the source. \\
5 authentication failure & The cryptographic message digest does not match the MAC. \\
6 unsynchronized & The server is not synchronized to a valid source. \\
7 bad header data & One or more header fields are invalid. \\
\hline
\end{tabular}
\label{packet_error_checks}
\caption{Packet Error Checks}
\end{table}

Processing continues by copying the packet variables to the peer
variables as shown in Figure 21. An example is described by the
packet() routine in Appendix A.5.1.1. The receive() routine
implements tests 1-5 in Figure 22; the packet() routine implements
tests 6-7. If errors are found, the packet is discarded and the peer
process exits.

The on-wire protocol calculates the clock offset theta and round-trip
delay delta from the four most recent timestamps as described in
Section 8. While it is, in principle, possible to do all
calculations except the first-order timestamp differences in fixed-
point arithmetic, it is much easier to convert the first-order
differences to floating doubles and do the remaining calculations in
that arithmetic, and this will be assumed in the following
description.

Next, the 8-bit p.reach shift register in the poll process described
in Section 13 is used to determine whether the server is reachable
and the data are fresh. The register is shifted left by one bit when
a packet is sent and the rightmost bit is set to zero. As valid
packets arrive, the rightmost bit is set to one. If the register
contains any nonzero bits, the server is considered reachable;
otherwise, it is unreachable. Since the peer poll interval might
have changed since the last packet, the host poll interval is
reviewed. An example is provided by the poll\_update() routine in
Appendix A.5.7.2.

The dispersion statistic epsilon(t) represents the maximum error due
to the frequency tolerance and time since the last packet was sent.
It is initialized

$$
\epsilon(t_0) = r.rho + s.rho + \Phi \cdot (T4 - T1)
$$

when the measurement is made at t\_0 according to the seconds counter.
Here, r.rho is the packet precision described in Section 7.3 and
s.rho is the system precision described in Section 11.1, both
expressed in seconds. These terms are necessary to account for the
uncertainty in reading the system clock in both the server and the
client.

The dispersion then grows at constant rate PHI; in other words, at
time t, epsilon(t) = epsilon(t\_0) + PHI * (t-t\_0). With the default
value PHI = 15 ppm, this amounts to about 1.3 s per day. With this
understanding, the argument t will be dropped and the dispersion
represented simply as epsilon. The remaining statistics are computed
by the clock filter algorithm described in the next section.
