\chapter{Poll Process}

Each association supports a poll process that runs at regular
intervals to construct and send packets in symmetric, client, and
broadcast server associations.  It runs continuously, whether or not
servers are reachable in order to manage the clock filter and reach
register.

\section{Poll Process Variables}

Figure 29 summarizes the common names, formula names, and a short
description of the poll process variables (lowercase) and parameters
(uppercase).  Unless noted otherwise, all variables have assumed
prefix p.

\begin{table}[htb]
\center
\begin{tabular}{c | c | c}
Name    & Formula & Description        \\
\hline
\hline
hpoll   & hpoll   & host poll exponent \\
last    & last    & last poll time     \\
next    & next    & next poll time     \\
reach   & reach   & reach register     \\
unreach & unreach & unreach counter    \\
UNREACH & 24      & unreach limit      \\
BCOUNT  & 8       & burst count        \\
BURST   & flag    & burst enable       \\
IBURST  & flag    & iburst enable      \\
\hline
\end{tabular}
\label{poll_process_variables_and_parameters}
\caption{Poll Process Variables and Parameters}
\end{table}

The poll process variables are allocated in the association data
structure along with the peer process variables.  The following is a
detailed description of the variables.  The parameters will be called
out in the following text.

\begin{description}

  \item[hpoll] signed integer representing the poll exponent, in $ \log_2 $ seconds

  \item[last] integer representing the seconds counter when the most recent
    packet was sent

  \item[next] integer representing the seconds counter when the next packet
    is to be sent

  \item[reach] 8-bit integer shift register shared by the peer and poll
    processes

  \item[unreach] integer representing the number of seconds the server has
    been unreachable

\end{description}

\section{Poll Process Operations}

As described previously, once each second the clock-adjust process is
called.  This routine calls the poll routine for each association in
turn.  If the time for the next poll message is greater than the
seconds counter, the routine returns immediately.  Symmetric (modes
1, 2), client (mode 3), and broadcast server (mode 5) associations
routinely send packets.  A broadcast client (mode 6) association runs
the routine to update the reach and unreach variables, but does not
send packets.  The poll process calls the transmit process to send a
packet.  If in a burst (burst > 0), nothing further is done except
call the poll update routine to set the next poll interval.
  If not in a burst, the reach variable is shifted left by one bit,
with zero replacing the rightmost bit.  If the server has not been
heard for the last three poll intervals, the clock filter routine is
called to increase the dispersion.  An example is shown in
Appendix A.5.7.3.

If the BURST flag is lit and the server is reachable and a valid
source of synchronization is available, the client sends a burst of
BCOUNT (8) packets at each poll interval.  The interval between
packets in the burst is two seconds.  This is useful to accurately
measure jitter with long poll intervals.  If the IBURST flag is lit
and this is the first packet sent when the server has been
unreachable, the client sends a burst.  This is useful to quickly
reduce the synchronization distance below the distance threshold and
synchronize the clock.

If the P\_MANY flag is lit in the p.flags word of the association,
this is a manycast client association.  Manycast client associations
send client mode packets to designated multicast group addresses at
MINPOLL intervals.  The association starts out with a TTL of 1.  If
by the time of the next poll there are fewer than MINCLOCK servers
have been mobilized, the ttl is increased by one.  If the ttl reaches
the limit TTLMAX, without finding MINCLOCK servers, the poll interval
increases until reaching BEACON, when it starts over from the
beginning.

The poll() routine includes a feature that backs off the poll
interval if the server becomes unreachable.  If reach is nonzero, the
server is reachable and unreach is set to zero; otherwise, unreach is
incremented by one for each poll to the maximum UNREACH.  Thereafter
for each poll hpoll is increased by one, which doubles the poll
interval up to the maximum MAXPOLL determined by the poll\_update()
routine.  When the server again becomes reachable, unreach is set to
zero, hpoll is reset to the tc system variable, and operation resumes
normally.

A packet is sent by the transmit process.  Some header values are
copied from the peer variables left by a previous packet and others
from the system variables.  Figure 30 shows which values are copied
to each header field.  In those implementations, using floating
double data types for root delay and root dispersion, these must be
converted to NTP short format.  All other fields are either copied
intact from peer and system variables or struck as a timestamp from
the system clock.

\begin{table}[htb]
\center
\begin{tabular}{c | c | c}
Packet Variable & <-- & Variable \\
\hline
\hline
x.leap      & <-- & s.leap      \\
x.version   & <-- & s.version   \\
x.mode      & <-- & s.mode      \\
x.stratum   & <-- & s.stratum   \\
x.poll      & <-- & s.poll      \\
x.precision & <-- & s.precision \\
x.rootdelay & <-- & s.rootdelay \\
x.rootdisp  & <-- & s.rootdisp  \\
x.refid     & <-- & s.refid     \\
x.reftime   & <-- & s.reftime   \\
x.org       & <-- & p.xmt       \\
x.rec       & <-- & p.dst       \\
x.xmt       & <-- & clock       \\
x.keyid     & <-- & p.keyid     \\
x.digest    & <-- & md5 digest  \\
\hline
\end{tabular}
\label{xmit_packet_packet_header}
\caption{xmit\_packet Packet Header}
\end{table}

The poll update routine is called when a valid packet is received and
immediately after a poll message has been sent.  If in a burst, the
poll interval is fixed at 2 s; otherwise, the host poll exponent
hpoll is set to the minimum of ppoll from the last packet received
and hpoll from the poll routine, but not less than MINPOLL or greater
than MAXPOLL.  Thus, the clock discipline can be oversampled but not
undersampled.  This is necessary to preserve subnet dynamic behavior
and protect against protocol errors.

The poll exponent is converted to an interval, which, when added to
the last poll time variable, determines the value of the next poll
time variable.  Finally, the last poll time variable is set to the
current seconds counter.
